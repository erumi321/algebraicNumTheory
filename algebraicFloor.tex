\documentclass[11pt]{article}

\usepackage[utf8]{inputenc}
\usepackage[T1]{fontenc}
\usepackage{geometry}
\usepackage{mathtools}
\usepackage{amsfonts}
\usepackage{amsmath}
\usepackage{amssymb}
\usepackage{mathptmx}
\usepackage[makeroom]{cancel}
\allowdisplaybreaks

\renewcommand*\contentsname{Table of Contents}
\title{Algebraic Definitions of Number Theory Functions}
\author{Eli Ruminer}
\date{}
\DeclarePairedDelimiter\ceil{\lceil}{\rceil}
\DeclarePairedDelimiter\floor{\lfloor}{\rfloor}
\DeclarePairedDelimiter\round{\lfloor}{\rceil}
\DeclarePairedDelimiter\fractional{\{}{\}}
\DeclareMathOperator{\hp}{\frac{\pi}{2}}
\newcommand{\Mod}[2]{\ #1\ \mathrm{mod}\ #2}
\newcommand{\fractionalAlg}{\frac{\cot^{\mathrm{-1}}(\cot(x\pi))}{\pi}}
\begin{document}
\begin{titlepage}
\clearpage\maketitle
\thispagestyle{empty}
\end{titlepage}

\tableofcontents
\newpage

\section{Introduction}
This paper seeks to show a method of defining a floor function in algebraic terms and thus defining other number theory functions algebraicly from knowing floor.

\section{Floor}
We start by defining the fractional operator, which gets the decimal part of x, and removes any whole part to be \[\fractional*{x}\]
Where\[0\leq\fractional*{x}\leq1\]
From this we can define the floor function, \(\floor*{x}\) as
\begin{gather*}
\floor*{x}=x-\fractional*{x}
\end{gather*}
To define this algebraically we first must find a function which finds \(\fractional*{x}\) to be able to subtract it. This can be found by using the function
\[\operatorname{a}(x)=\tan^{\mathrm{-1}}(\tan(x))\]
This creates a periodic function increasing linearly from \(-\hp\) to \(\hp\) over \([n\pi-\hp,n\pi+\hp]\) where \(n\in\mathbb{Z}\). We must first decrease the period from \(\pi\) to 1, and range from \(\pi\) to 1, giving us
\[\operatorname{b}(x)=\frac{a(x\pi)}{\pi}=\frac{\tan^{\mathrm{-1}}(\tan(x\pi))}{\pi}\]
The function then must be phase shifted 0.5 to "start" each period at an integer, and shifted up so that it is always positive giving us

\begin{align*}
\fractional*{x}=\operatorname{b}(x-\frac{1}{2}) + \frac{1}{2} & =\frac{\tan^{\mathrm{-1}}(\tan(x\pi-\frac{\pi}{2}))}{\pi}+\frac{1}{2}\\
& = \fractionalAlg
\end{align*}

We then plug the new definition of \(\fractional*{x}\) into the original floor equation to find that
\[\floor*{x}=x-\fractionalAlg\]
Due to using tangent, there are undefined asymptotes at the whole numbers, thus floor cannot be defined truly algebraically, but must still have a piecewise definition of
\begin{gather*}
\floor*{x}=\begin{cases}
x & \text{if } x \in \mathbb{Z}\\
x-\fractionalAlg & \text{if } x \in \mathbb{R}\setminus\mathbb{Z}\\
\end{cases}
\end{gather*}
Unless explicitly stated the non-piecewise definition will be used for the rest of the paper.

\section{Ceiling and Round}
From floor, we can define the other 2 types of rounding functions: Ceiling and Round.
\subsection{Ceiling}
Ceiling (ceil) always rounds up and can be defined as
\begin{gather*}
\ceil*{x}=\begin{cases}
x & \text{if } x \in \mathbb{Z}\\
x-\fractional*{x}+1 & \text{if } x \in \mathbb{R}\setminus\mathbb{Z}
\end{cases}
\end{gather*}
which one will notices shares \(x-\fractional*{x}\) with \(\floor*{x}\) and thus
\begin{equation*}
\begin{split}
\ceil*{x}&=\begin{cases}
x & \text{if } x \in \mathbb{Z}\\
\floor*{x}+1 & \text{if } x \in \mathbb{R}\setminus\mathbb{Z}
\end{cases}\\
&=\begin{cases}
x & \text{if } x \in \mathbb{Z}\\
x-\fractionalAlg & \text{if } x \in \mathbb{R}\setminus\mathbb{Z}\\
\end{cases}
\end{split}
\end{equation*}
\subsection{Round}
Round is defined as
\begin{equation*}
\begin{split}
\round*{x}&=\begin{cases}
x - \fractional*{x} & \text{if } \fractional*{x} < 0.5\\
x-\fractional*{x}+1 & \text{if } \fractional*{x} \geq 0.5
\end{cases}\\
&=\begin{cases}
\floor*{x} & \text{if } \fractional*{x} < 0.5\\
\ceil*{x} & \text{if } \fractional*{x} \geq 0.5
\end{cases}
\end{split}
\end{equation*}
One notices that \(\round*{x}=\floor*{x+0.5}\) thus
\begin{equation*}
\round*{x}=x-\frac{\tan^{\mathrm{-1}}(\tan(x\pi))}{\pi}=\begin{cases}
x & \text{if } x \in \mathbb{Z}\\
x-\frac{\tan^{\mathrm{-1}}(\tan(x\pi))}{\pi} & \text{if } x \in \mathbb{R}\setminus\mathbb{Z}\\
\end{cases}
\end{equation*}

\section{Mod}
We start by defining the function \(\Mod{a}{b}\) as
\begin{align*}
\Mod{a}{b} &=n \textrm{ where } a \equiv n (\textrm{mod} b)\\
&=a-b\floor*{\frac{a}{b}}
\end{align*}
which substituting the previously defined definition of floor results in
\begin{equation*}
\begin{split}
\Mod{a}{b}&=a-b(\frac{a}{b}-\frac{\cot^{\mathrm{-1}}(\cot(\frac{a\pi}{b}))}{\pi})
\\&=\frac{b}{\pi}\cot^{\mathrm{-1}}(\cot(\frac{a\pi}{b}))
\end{split}
\end{equation*}
\section{Proof}
\begin{equation}
\frac{d}{dx} \floor{x}=\frac{d}{dx} (x-\fractionalAlg)=1-1=0
\end{equation}
By the definition of \(\floor{x}\), when \(x\in\mathbb{Z},\ \floor{x}=x\), and that \(\frac{d}{dx}\floor{x}=0\), \(\floor{x}\) is constant between integers, and jumps to each integer at that integer \(\blacksquare\)
\end{document}